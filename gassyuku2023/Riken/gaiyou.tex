\documentclass[uplatex,a4paper, 11pt]{jsarticle}


% % 数式
% \usepackage{amsmath,amsthm,amssymb}
% \usepackage{bm}
% % 画像
% \usepackage[dvipdfmx]{graphicx, color}

% \usepackage{multirow}
% \usepackage{wrapfig}
\usepackage{ascmac}

% \graphicspath{{../../_Figures//}{../../_Figures/PhaseSeparation/}}


% %
% %数式の途中で改行
% \allowdisplaybreaks[3]
% %
% %微分関連のマクロ
% %
% \newcommand{\diff}{\mathrm d}
% \newcommand{\difd}[2]{\dfrac{\diff #1}{\diff #2}}
% \newcommand{\difp}[2]{\dfrac{\partial #1}{\partial #2}}
% \newcommand{\difdd}[2]{\dfrac{\diff^2 #1}{\diff #2^2}}
% \newcommand{\difpp}[2]{\dfrac{\partial^2 #1}{\partial #2^2}}

% ページ設定
\pagestyle{empty}
% 高さの設定
\usepackage[top=20truemm,bottom=20truemm,left=20truemm,right=20truemm]{geometry}

%
\renewcommand{\baselinestretch}{0.98} % 全体の行間調整

% \usepackage[dvipdfmx]{hyperref}
% \usepackage{pxjahyper}
% \hypersetup{%
% bookmarksnumbered=true,%
% colorlinks=true,%
% setpagesize=false,%
% pdftitle={化学系企業で物理と化学の狭間で考えてきたこと},%
% pdfauthor={佐々木裕},%
% pdfsubject={},%
% pdfkeywords={物理、化学、思考ベースでの相互理解}}

\title{化学系企業で物理と化学の狭間で考えてきたこと\\ \Large{~コウモリ研究者の戯言~}}
% \subtitle{~コウモリ研究者の戯言~}
\author{佐々木裕\thanks{東亞合成株式会社}}
\date{}%{2021, Nov. 25-26}

\begin{document}

\maketitle
\vspace{-2mm}
% \section*{概要}
近年、AI 関連技術としての機械学習やシミュレーションのベースとなる計算機資源の拡充や方法論の革新的な進展が、
世界的な規模で加速度的に進展していることは言うまでもありません。
一方、産業界のみならず学術界においても日本の凋落が顕著になってきていると感じている方は、少なからずいらっしゃるでしょう。
また、東日本大震災、各地豪雨による土砂災害、COVID-19 による医療崩壊等々の災厄に対する後手に回ったとしか言えない対応の不手際の多さには目に余るものがあります。
このような日本のシステム的な疲弊を放置すれば、先進国からの周回遅れどころでは済まなくなります。
規模の小さい化学系企業に所属する一介の研究者である話者の立場からは、国家のあるべき姿のような大所高所に立った俯瞰的な意見を述べるようなことはできません。
しかしながら、上記のような周回遅れは材料開発の現場においても顕著に生じていると痛感しています。

話者は、大学時代の合成化学の知見をベースにして化学系企業での研究・開発生活をスタートし、
必要に迫られて学んできた評価技術の背景にある物理的な思考の重要性に気づき、
そこから、レオロジーやシミュレーションのベースとなるソフトマター物理へと軸足を移してきました。
その過程で幅広い分野の研究者の方とのふれあいを深める中で、化学系の人間としての過去の自身に欠けていたものが、
物理系では当たり前の考え方である「自然現象の背後にあるユニバーサリティーの理解と適正なレベルでのモデル化」だと感じてきました。

近年の CAX(Computer Aided something) の長足の進歩を実際の材料の開発へとつなげるためには、
化学系の研究者が試行錯誤ベースで実際に物質を合成することを欠かすことはできません。
この実材料の創成の過程において、化学系の研究者が物理側から提案された理論的な成果を盲目的に受け入れるのではなく、
その背景を理解する必要があります。
また、物理分野の人たちがモデルを構築する際に、物質に内在する多様性を実感しつつモデルの条件設定を適切に行っていただくことも重要です。
結局、物理と化学のあいだで、思考(志向)ベースまで含めた互いの方法論についての相互理解を深めていくことこそが重要なのだと考えるに至りました。
そして、それを次世代の若者につなげていくためには、目先の成果を急かすことなく広範な視野を習得できるような環境づくりに務めることが必要だと痛感しています。
\vspace{3mm}
\begin{boxnote}
    \vspace{-1mm}
    ここでお話させていただく内容を、以下に簡単にまとめました。
    \vspace{-1mm}
\begin{enumerate}
    \item かんたんな自己紹介
    \begin{itemize}
        \item 研究・開発歴、そして、アプローチ方法の変遷
        \item 私のモットー
            % \begin{itemize}
            %     \item 急がば回れ
            %     \item 備えよ常に
            % \end{itemize}
    \end{itemize}
    \item (少なくとも化学系企業で)開発にありがちな状態
    \begin{itemize}
        \item 教科書的なものの背後にある物理的、数学的な思想を理解することからの逃避
        \item 数式や物理モデルの盲目的な受認によるデータの処理
        \item 客観的な視点に基づく独立事象と従属事象の切り分けの放棄
    \end{itemize}
    \item あるべき状態
    \begin{itemize}
        \item 物理的な思考による事象の成り立ちの理解およびモデル化への道すじを共有
        \item 目的を明確にし、適切な次元、スケール及び時間軸で議論を行う
        % \item 物質の多様性を前提とした化学的な方法論の整理と適正なモデル化への挑戦
        % \item 物理及び化学双方の方法論についての相互理解の深化
    \end{itemize}
    \item まとめに代えてのディスカッション
\end{enumerate}
\end{boxnote}




% 物事には仕組みがあることを忘れない。

% 切り取りではなく全体を見通すこと

% 仕組みを作り出すには論理の流れが必要

% 先達の仕事には理屈の裏付けがある

% ときには、それは後付のときもあるが。

% QCの功罪

% よくあるまずいこと

% 積んだブロックを壊すことなく無理やり上に積み上げる。

% 急いで結果を出そうとする

% 論理は再構築することで磨かれる。

% 自然の多様性とその裏にある類似性

% ソフトマター物理でよく見られる「大いなる同一視」

% 物理でよくあるユニバーサリティーへの期待

\end{document}